% Template for PLoS
% Version 1.0 January 2009
%
% To compile to pdf, run:
% latex plos.template
% bibtex plos.template
% latex plos.template
% latex plos.template
% dvipdf plos.template

\documentclass[10pt]{article}

% amsmath package, useful for mathematical formulas
\usepackage{amsmath}
% amssymb package, useful for mathematical symbols
\usepackage{amssymb}

% graphicx package, useful for including eps and pdf graphics
% include graphics with the command \includegraphics
\usepackage{graphicx}

% cite package, to clean up citations in the main text. Do not remove.
\usepackage{cite}

\usepackage{color}

% Use doublespacing - comment out for single spacing
\usepackage{setspace}
\doublespacing


% Text layout
\topmargin 0.0cm
\oddsidemargin 0.5cm
\evensidemargin 0.5cm
\textwidth 16cm
\textheight 21cm

% Bold the 'Figure #' in the caption and separate it with a period
% Captions will be left justified
\usepackage[labelfont=bf,labelsep=period,justification=raggedright]{caption}

% Use the PLoS provided bibtex style
\bibliographystyle{plos2009}

% Remove brackets from numbering in List of References
\makeatletter
\renewcommand{\@biblabel}[1]{\quad#1.}
\makeatother


% Leave date blank
\date{}

\pagestyle{myheadings}
%% ** EDIT HERE **


%% ** EDIT HERE **
%% PLEASE INCLUDE ALL MACROS BELOW

\newcommand{\HMIC}{{IC}$_{50}$}

%% END MACROS SECTION

\begin{document}

% Title must be 150 characters or less
\begin{flushleft}
{\Large
\textbf{{IDEPI}: rapid prediction of HIV-1 antibody epitopes leveraging mutual information and support vector machines}
}
% Insert Author names, affiliations and corresponding author email.
\\
N Lance Hepler$^{1,\ast}$,
Konrad Scheffler$^{2}$,
Douglas D Richman$^{2,3}$,
Dennis R Burton$^{4,5}$,
Sergei L Kosakovsky Pond$^{2}$
\\
\bf{1} Interdisciplinary Bioinformatics and Systems Biology Program, University of California San Diego, La Jolla, CA, USA
\\
\bf{2} Department of Medicine, University of California San Diego, La Jolla, CA, USA
\\
\bf{3} San Diego Veterans Affairs Healthcare System, San Diego, CA, USA
\\
\bf{4} The Scripps Research Institute, La Jolla, CA, USA
\\
\bf{5} Ragon Institute of {MGH}, {MIT}, and Harvard, Boston, MA, USA
\\
$\ast$ E-mail: Corresponding nhepler@ucsd.edu
\end{flushleft}

% Please keep the abstract between 250 and 300 words
\section*{Abstract}

% Please keep the Author Summary between 150 and 200 words
% Use first person. PLoS ONE authors please skip this step.
% Author Summary not valid for PLoS ONE submissions.
\section*{Author Summary}

\section*{Introduction}
Introduced in the late 20th century,
the human immunodeficiency virus type 1 (HIV-1) has infected approximately 60 million people
and caused approximately 25 million deaths.
The advent of highly active antiretroviral therapy ({HAART}) has contributed greatly to the fight against the HIV-1 epidemic.
Unfortunately, HAART is neither curative, nor uniformly effective, nor widely available.
These facts have reinforced the need for a preventative and potentially curative vaccine.
Unfortunately, such a vaccine has remained elusive.
This is largely due to the combination of HIV-1’s mutational speed,
its proclivity for genomic recombination events,
and the unique and heterogeneous structure of its envelope protein (env),
all of which render HIV-1 infections resistant to immunological attack.
Recently, the discovery of several broadly neutralizing antibodies (bnAbs) has suggested new routes toward the invention of an effective vaccine.

In order to effectively mimic the activity of these bnAbs,
we must first characterize their active surface, known as an epitope.
This is an open and difficult problem, largely due to the structural nature of nAb epitopes.
There are several techniques for mapping nAb epitopes, but they often have caveats that limit or restrict their generalizability.
For example, one such technique is a reverse genetics approach known as “alanine screening”,
wherein each and every position of the env protein is individually mutated to an alanine,
which is chemically inert (relative to other amino acids) while remaining chiral (unlike glycine).
The resulting constructs are then tested for binding activity.
Given that there are 856 amino acid positions in the HIV-1 reference sequence ({HXB2}),
such a technique is obviously labor and resource intensive.
Additionally, it is hampered by biological caveats: some positions cannot be realistically mutated in vivo,
there are fitness constraints that must be accounted for;
the constructs used to test binding are often monomeric forms of the env protein (which is trimeric \em{in vivo}),
and as such the results are not guaranteed to be representative of biological reality.
Other biochemical and structural methods suffer similar caveats:
x-ray crystallography often requires the use of chimeric constructs to facilitate crystallization.
Computational approaches to epitope mapping offer the promise of rapid results,
and paradoxically more biologically realistic results.

<talk about other computational approaches, e.g. CHAVI, possibly miguel, and others>
<talk about limitations in their approaches, either due to speed or unsimulatability> I propose one such method here, and demonstrate its efficacy on a variety of previously-characterized bnAb epitopes, and validate its performance across a variety of simulated inputs.

For a given neutralizing agent,
the IDEPI pipeline uses predictive modeling to identify common compensatory mutations between sequences of high sensitivity and low sensitivity.
We hypothesize that common compensatory mutations will often lie within the epitope.

% Results and Discussion can be combined.
\section*{Results}

\subsection*{Subsection 1}

\subsection*{Subsection 2}

\section*{Discussion}

% You may title this section "Methods" or "Models".
% "Models" is not a valid title for PLoS ONE authors. However, PLoS ONE
% authors may use "Analysis"
\section*{Materials and Methods}
Data were obtained using a Monogram neutralization assay,
which measures the half-maximal inhibitory concentration (\HMIC)
of a neutralizing agent (monoclonal antibody or polyclonal sera) in a model system composed of a chimeric HIV-1 construct expressing a known envelope sequence.
This assay was performed for 64 distinct neutralizing agents against 9 to 421 envelope sequences per agent (median 116).

The IDEPI pipeline is composed of the following stages:
1) collate the \HMIC values and corresponding envelope sequences for our neutralizing agent of interest,
2) produce a multiple sequence alignment from the collated sequences using the {HMMER} multiple sequence alignment tool,
3) binarize the data by thresholding the \HMIC values (at some natural or biologically-relevant value)
and transforming each column of the MSA into a vector of 23 binary states representing the presence (1) or absence (0) of a given amino-acid at any given position,
4) partition the sequences into $k$ bins,
5) for each bin $B_i$,
5a) set $B_i$ aside to use as test data,
while using the remaining $k-1$ bins to learn a predictive model,
5b) test the performance of the predictive model on the test data in bin $B_i$,
6) collate and present the predictive model performance and the features chosen by the model as estimations of confidence and inferred epitope positions, respectively.

% Do NOT remove this, even if you are not including acknowledgments
\section*{Acknowledgments}


%\section*{References}
% The bibtex filename
\bibliography{plos}

\section*{Figure Legends}
%\begin{figure}[!ht]
%\begin{center}
%%\includegraphics[width=4in]{figure_name.2.eps}
%\end{center}
%\caption{
%{\bf Bold the first sentence.}  Rest of figure 2  caption.  Caption
%should be left justified, as specified by the options to the caption
%package.
%}
%\label{Figure_label}
%\end{figure}


\section*{Tables}
%\begin{table}[!ht]
%\caption{
%\bf{Table title}}
%\begin{tabular}{|c|c|c|}
%table information
%\end{tabular}
%\begin{flushleft}Table caption
%\end{flushleft}
%\label{tab:label}
% \end{table}

\end{document}
